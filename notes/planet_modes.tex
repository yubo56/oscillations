    \documentclass[11pt,
        usenames, % allows access to some tikz colors
        twocolumn,
        landscape,
        dvipsnames % more colors: https://en.wikibooks.org/wiki/LaTeX/Colors
    ]{article}
    \usepackage{
        amsmath,
        amssymb,
        fouriernc, % fourier font w/ new century book
        fancyhdr, % page styling
        lastpage, % footer fanciness
        hyperref, % various links
        setspace, % line spacing
        amsthm, % newtheorem and proof environment
        mathtools, % \Aboxed for boxing inside aligns, among others
        float, % Allow [H] figure env alignment
        enumerate, % Allow custom enumerate numbering
        graphicx, % allow includegraphics with more filetypes
        wasysym, % \smiley!
        upgreek, % \upmu for \mum macro
        listings, % writing TrueType fonts and including code prettily
        tikz, % drawing things
        booktabs, % \bottomrule instead of hline apparently
        xcolor, % colored text
        cancel % can cancel things out!
    }
    \usepackage[margin=0.5in]{geometry} % page geometry
    \usepackage[
        labelfont=bf, % caption names are labeled in bold
        font=scriptsize % smaller font for captions
    ]{caption}
    \usepackage[font=scriptsize]{subcaption} % subfigures

    \newcommand*{\scinot}[2]{#1\times10^{#2}}
    \newcommand*{\dotp}[2]{\left<#1\,\middle|\,#2\right>}
    \newcommand*{\rd}[2]{\frac{\mathrm{d}#1}{\mathrm{d}#2}}
    \newcommand*{\pd}[2]{\frac{\partial#1}{\partial#2}}
    \newcommand*{\rdil}[2]{\mathrm{d}#1 / \mathrm{d}#2}
    \newcommand*{\pdil}[2]{\partial#1 / \partial#2}
    \newcommand*{\rtd}[2]{\frac{\mathrm{d}^2#1}{\mathrm{d}#2^2}}
    \newcommand*{\ptd}[2]{\frac{\partial^2 #1}{\partial#2^2}}
    \newcommand*{\md}[2]{\frac{\mathrm{D}#1}{\mathrm{D}#2}}
    \newcommand*{\pvec}[1]{\vec{#1}^{\,\prime}}
    \newcommand*{\svec}[1]{\vec{#1}\;\!}
    \newcommand*{\bm}[1]{\boldsymbol{\mathbf{#1}}}
    \newcommand*{\uv}[1]{\hat{\bm{#1}}}
    \newcommand*{\ang}[0]{\;\text{\AA}}
    \newcommand*{\mum}[0]{\;\upmu \mathrm{m}}
    \newcommand*{\at}[1]{\left.#1\right|}
    \newcommand*{\bra}[1]{\left<#1\right|}
    \newcommand*{\ket}[1]{\left|#1\right>}
    \newcommand*{\abs}[1]{\left|#1\right|}
    \newcommand*{\ev}[1]{\left\langle#1\right\rangle}
    \newcommand*{\p}[1]{\left(#1\right)}
    \newcommand*{\s}[1]{\left[#1\right]}
    \newcommand*{\z}[1]{\left\{#1\right\}}

    \newtheorem{theorem}{Theorem}[section]

    \let\Re\undefined
    \let\Im\undefined
    \DeclareMathOperator{\Res}{Res}
    \DeclareMathOperator{\Re}{Re}
    \DeclareMathOperator{\Im}{Im}
    \DeclareMathOperator{\Log}{Log}
    \DeclareMathOperator{\Arg}{Arg}
    \DeclareMathOperator{\Tr}{Tr}
    \DeclareMathOperator{\E}{E}
    \DeclareMathOperator{\Var}{Var}
    \DeclareMathOperator*{\argmin}{argmin}
    \DeclareMathOperator*{\argmax}{argmax}
    \DeclareMathOperator{\sgn}{sgn}
    \DeclareMathOperator{\diag}{diag\;}

    \colorlet{Corr}{red}

    % \everymath{\displaystyle} % biggify limits of inline sums and integrals
    \tikzstyle{circ} % usage: \node[circ, placement] (label) {text};
        = [draw, circle, fill=white, node distance=3cm, minimum height=2em]
    \definecolor{commentgreen}{rgb}{0,0.6,0}
    \lstset{
        basicstyle=\ttfamily\footnotesize,
        frame=single,
        numbers=left,
        showstringspaces=false,
        keywordstyle=\color{blue},
        stringstyle=\color{purple},
        commentstyle=\color{commentgreen},
        morecomment=[l][\color{magenta}]{\#}
    }

\begin{document}

\section{11/29/22}

We follow Appendicies B-C of Le Bihan \& Burrows 2013. Note that we can seek two
independent solutions to the ODEs because it is not a simple Sturm-Liouville
form problem: the structure is quite complex, so we aren't guaranteed simple
roots. Perhaps it will be fun to try and do the properties of Sturm-Liouville
proofs again someday\dots

Update: we are able to get seemingly correct oscillation modes with dummy
stellar profiles, but we don't have the correct non-singular solution near the
core. Due to this, the matrix determinants in the core-surface solution matching
are large, and so we suffer from some loss of numerical precision.

\section{12/07/22}

\subsection{Polytrope Solution}

Let's review how to obtain the structure of a polytrope. The equations of
hydrostatic equilibrum are:
\begin{align}
    \rd{P}{r} &= -\frac{GM_r}{r^2}\rho,\\
    \rd{M_r}{r} &= 4\pi r^2\rho,\\
    P &= K\rho^\Gamma = K\rho^{1 + 1/n}.
\end{align}
Here, $K$ is constant. This can easily be rearranged to obtain
\begin{equation}
    \rd{}{r}\p{\frac{r^2}{\rho}\rd{P}{r}} = \rd{}{r}\p{-GM_r} = -G4\pi r^2\rho.
\end{equation}
Then, since $\pdil{P}{r} = K\rho^{\Gamma - 1}\Gamma \rdil{p}{r}$, we have that
\begin{equation}
    \frac{1}{r^2}\rd{}{r}\p{r^2 K\rho^{\Gamma - 2}\Gamma \rd{\rho}{r}}
        = -G 4\pi \rho.
\end{equation}
Now, let's guess that $\rho(r) = \rho_c \theta^p(r)$, where $p$ must be
chosen such that $\rho^{\Gamma - 2}\rdil{\rho}{r} \propto \rdil{\theta}{r}$.
This can be seen to yield that $p = 1 / (\Gamma - 1) = n$, so then
\begin{align}
    \frac{1}{r^2}\rd{}{r}\p{\Gamma r^2Kn\rho_c^{\Gamma - 1}\rd{\theta}{r}}
        &= -G 4\pi \rho_c \theta^n,\\
    \frac{1}{r^2}\rd{}{r}\p{r^2\rd{\theta}{r}}
        &= -\theta^n \times \p{\frac{G4\pi \rho_c^{1 - 1/n}}{K\p{n + 1}}}
        \equiv -\theta^n \frac{1}{\alpha^2}.
\end{align}
Let's define $\xi \equiv r/\alpha$, then the Lane Emden-Equation reduces to (let
primes denote $\rdil{}{\xi}$)
\begin{equation}
    \theta'' + \frac{2\theta'}{\xi} +\theta^n = 0.
\end{equation}
Note that the ICs for $\theta$ are
\begin{align}
    \theta(0) &= 1, & \theta'(0) &= 0.
\end{align}
It's clear that $\theta'(0) = 0$ is required for the equation to be non-singular.
Let's guess that $\theta'(\xi \ll 1) \propto \xi$, then upon inspection we find
that
\begin{equation}
    \theta'(\xi \ll 1) = -\xi / 3
\end{equation}
is correct, which is useful later. In addition, $\theta(\xi_1) = 0$ exists if $n
< 5$, so $\xi_1$ is the outer boundary.

From here, we seek to solve for all of the quantities in closed form; first the
stellar properties, then the relevant dimensionless quantities for the
oscillation equations. First, it is clear that our natural parameterization of
the stellar structure above relies on $K$ and $\rho_c$, while we want to express
these in terms of $M$ and $R$.
\begin{align}
    \alpha &= \p{\frac{K\p{n + 1}}{4\pi G \rho_c^{1 - 1/n}}}^{1/2},\\
    C &\equiv \p{\frac{K\p{n + 1}}{4\pi G}}^{1/2}
        = \alpha \rho^{\frac{n - 1}{2n}},\\
    R &= \alpha \xi_1\nonumber\\
        &\equiv C^{1/2}\rho_c^{\frac{1 - n}{2n}} \xi_1,\\
    M &= \int\limits^R 4\pi r^2 \rho(r)\;\mathrm{d}r\nonumber\\
        &= 4\pi \rho_c \alpha^3 \int\limits^{\xi_1} \xi^2
            \theta^n(\xi)\;\mathrm{d}\xi\nonumber\\
        &= 4\pi \rho_c \alpha^3 \int\limits^{\xi_1}
            -\rd{}{\xi}\p{\xi^2\rd{\theta}{\xi}}\;\mathrm{d}\xi\nonumber\\
        &= 4\pi \rho_c^{\frac{3 - n}{2n}} C^{3/2}
            \s{-\p{\xi^2\rd{\theta}{\xi}}}_{\xi_1},\\
    M &= 4\pi \rho_c^{\frac{3 - n}{2n}}
            \p{\frac{R}{\rho_c^{(1 - n) / (2n)}\xi_1}}^3
            \s{-\p{\xi^2\rd{\theta}{\xi}}}_{\xi_1}\nonumber\\
        &= 4\pi \rho_c \p{\frac{R}{\xi_1}}^3
            \s{-\p{\xi^2\rd{\theta}{\xi}}}_{\xi_1}\nonumber\\
    \rho_c &= \p{\frac{M}{4\pi R^3/3}}
        \p{-\frac{\xi_1}{3\theta'(\xi_1)}},\\
    R &= \p{\frac{K\p{n + 1}}{4\pi G}}^{1/2}
            \rho_c^{\frac{1 - n}{2n}}\xi_1\nonumber\\
    K &= \frac{R^2}{\xi_1^2}\rho_c^{\frac{n - 1}{n}}\frac{4\pi G}{n + 1}.
\end{align}
As such, we have $\p{K, \rho_c}\leftrightarrow \p{M, R}$, so we use $K, \rho_c$
for simplicity. The rest follow:
\begin{align}
    \rho(\xi) &= \rho_c \theta^n(\xi),\\
    P(\xi) &= K\rho_c^\Gamma \theta^{n + 1}(\xi) \propto \theta^{n + 1}(\xi),\\
    M_r(\xi) &= \int\limits^{\xi \alpha} 4\pi r^2\rho\;\mathrm{d}r\nonumber\\
        &= 4\pi \alpha^3 \rho_c
            \int\limits^{\xi} \xi^2\theta^n(\xi)\;\mathrm{d}\xi\nonumber\\
        &= -4\pi \alpha^3 \rho_c \xi^2\theta'(\xi),\\
    g(\xi) &= \frac{GM_r}{(\xi\alpha)^2}\nonumber\\
        &= -4\pi G \alpha \rho_c \theta'(\xi),\\
    U(\xi) &= \frac{4\pi \rho (\xi\alpha)^3}{M_r}\nonumber\\
        &= -\frac{\theta^n \xi}{\theta'(\xi)}\\
    c_1 &= \textcolor{red}{\frac{r / R}{M_r / M}}\nonumber\\
        &= \frac{1}{(\xi / \xi_1)(\theta'(\xi) / \theta'(\xi_1))}.
\end{align}
Note that $U(\xi \to 1) = 3$ and $c_1(\xi \to 0) \propto -1 / \xi^2$.

There's a small point of contention about $c_s$ and $N$, since they depend on
the adiabatic index $\Gamma_1$ and not just the polytropic index $\Gamma$. It
takes a bit of digging, but Mullan \& Ulrich 1988 point out that $\Gamma_1 =
5/3$ is used in all of their models. Thus, we have the remaining quantities in
the Le Bihan formulation:
\begin{align}
    c_s^2(\xi) &= \Gamma_1\frac{P}{\rho} = \Gamma_1 K\rho_c^{1/n}\theta(\xi)
        ,\\
    N^2(\xi) &= g^2\p{\rd{\rho}{P} - \frac{1}{c_s^2}}\nonumber\\
        &= \p{\frac{1}{\Gamma} - \frac{1}{\Gamma_1}}
            g^2\frac{\rho}{P}\nonumber\\
    A^* &= \frac{N^2r}{g} = \frac{\rho gr}{P}\p{\frac{1}{\Gamma} -
            \frac{1}{\Gamma_1}}\nonumber\\
        &= \p{\frac{1}{\Gamma} - \frac{1}{\Gamma_1}}
            \p{-4\pi G \alpha \rho_c \theta'(\xi)}\p{\alpha \xi}
            \frac{1}{K\rho_c^{1/n}\theta(\xi)}\nonumber\\
        &= \p{\frac{1}{\Gamma} - \frac{1}{\Gamma_1}}
            \p{-4\pi G \rho_c^{1 - 1/n}}
                \alpha^2
            \frac{\xi \theta'(\xi)}{K\theta(\xi)}\nonumber\\
        &= -\p{\frac{1}{\Gamma} - \frac{1}{\Gamma_1}}
            \frac{(n + 1)\xi \theta'(\xi)}{\theta(\xi)}\nonumber\\
    V_g &= \frac{gr}{c_s^2}\nonumber\\
        &= A^* \frac{1 / \Gamma_1}{1 / \Gamma - 1 / \Gamma_1}\nonumber\\
        &= -\frac{1}{\Gamma_1} \frac{(n + 1)\xi \theta'(\xi)}{\theta(\xi)}
\end{align}

\subsection{Oscillation Equations}

\subsubsection{Physical Form}

Let's rederive the oscillation equations too, all the way through the
nondimensionalized form (we did this in Dong's class). We start with the Euler
Equations and Poisson equation
\begin{align}
    \pd{\rho}{t} + \vec{\nabla} \cdot \p{\rho \vec{v}} &= 0,\\
    \pd{\vec{u}}{t} + \p{\vec{u} \cdot \vec{\nabla}}\vec{u}
        + \frac{\vec{\nabla}P}{\rho} &= -\vec{\nabla} \Phi,\\
    \nabla^2 \Phi = 4\pi G\rho.
\end{align}
We expand $x = x_0 + \delta x$ for each of $x = P, \rho, \Phi$. Then we assume
an adiabatic EOS
\begin{equation}
    \frac{\Delta P}{\Delta \rho} = c_s^2,
\end{equation}
where $\Delta$ is the Lagrangian perturbation, and we have the useful relation
\begin{align}
    \delta \vec{v} &= \rd{\vec{\xi}}{t}, &
    \Delta x &= \delta x + \p{\vec{\xi} \cdot \vec{\nabla}}x.
\end{align}
equilibrium $\rdil{P_0}{r} = -\rho_0 g$. The goal is to take FTs and cast
everything into $\xi_r$, $\delta P$, and $\delta \Phi$ with only three equations
of motion.

Then we first tackle the density equation. There is no leading order term, so we
directly analyze the perturbative component. There are two forms that I think
might be useful:
\begin{align}
    \cancel{\pd{\rho_0}{t}} +
    \pd{(\delta \rho)}{t} + \vec{\nabla} \cdot \p{\rho_0 \delta \vec{u}}
        &= 0,\nonumber\\
    \delta \rho &= -\vec{\nabla} \cdot \p{\rho_0 \vec{\xi}},\\
    \rd{(\delta \rho)}{t} &= -\rho_0\p{\vec{\nabla} \cdot \p{\delta \vec{u}}}
        ,\nonumber\\
    \Delta \rho &= -\rho_0\p{\vec{\nabla} \cdot \vec{\xi}}.\label{eq:pert_dens}
\end{align}
Next, we tackle the momentum equation. The zeroth order term is just hydrostatic
equilibrium, corresponding to $\vec{\nabla}P_0 / \rho_0 = -\vec{\nabla} \Phi_0$,
which gives $g = \vec{\nabla} \Phi_0$. The first order term is
\begin{align}
    \pd{(\delta \vec{u})}{t} + \frac{\vec{\nabla} (\delta P)}{\rho_0}
        - \frac{(\delta \rho) \vec{\nabla}P_0}{\rho_0^2} &=
        -\vec{\nabla}\p{\delta \Phi},\nonumber\\
    \ptd{\vec{\xi}}{t} + \frac{\vec{\nabla}(\delta P)}{\rho_0}
        - \frac{\delta \rho \vec{\nabla}P_0}{\rho_0^2} &= \vec{\nabla}(\delta
        \Phi).\label{eq:pert_mom}
\end{align}
We go ahead and substitute $\partial_t \to i\omega$. The perpendicular and
parallel components of Eq.~\eqref{eq:pert_mom} are:
\begin{align}
    \hat{r} &: -\omega^2 \xi_r + \frac{1}{\rho_0}\pd{}{r}(\delta P)
         + \frac{\delta r}{\rho_0}g = -\pd{}{r}\delta \Phi,\\
    \perp &: -\omega^2 \vec{\xi}_\perp + \frac{\vec{\nabla}_\perp (\delta P)}{
        \rho_0} = -\vec{\nabla}\p{\delta\Phi}\nonumber\\
        &\; -\omega^2\vec{\xi}_\perp = -\vec{\nabla}\p{\frac{\delta P}{\rho_0}
            + \delta \Phi}.
\end{align}
Combining the perpendicular component with Eq.~\eqref{eq:pert_dens}, we can
rewrite (and substitute in $Y_{lm}$'s)
\begin{align}
    \Delta \rho &= -\rho_0\p{\frac{1}{r^2}\ptd{}{r}\p{r^2\xi_r}
            + \nabla_\perp^2 \p{\frac{\delta P}{\rho_0\omega^2} + \frac{\delta
            \Phi}{\omega^2}}},\nonumber\\
    -\frac{\Delta \rho}{\rho_0}
        &= \frac{1}{r^2}\pd{}{r}\p{r^2\xi_r}
            - \frac{l(l + 1)}{r^2\omega^2}\p{\frac{\delta P}{\rho_0}
                + \delta \Phi},\nonumber\\
    &= -\frac{\Delta P}{c_s^2\rho_0} = -\frac{\delta P}{c_s^2\rho_0}
            + \frac{\xi_r g}{c_s^2}.
\end{align}
To rewrite the radial component of the momentum equation, we first define $N^2$
and rewrite:
\begin{align}
    N^2 &\equiv g^2\p{\rd{\rho_0}{P_0} - \frac{1}{c_s^2}},\\
    \frac{N^2}{g^2} + \frac{1}{c_s^2} &= \frac{\rdil{\rho_0}{r}}{
        \rdil{P_0}{r}},\nonumber\\
    \delta \rho&= \Delta \rho - \xi_r \pd{\rho_0}{r}\nonumber\\
        &= \frac{\Delta P}{c_s^2} - \xi_r\p{\frac{N^2}{g^2} + \frac{1}{c_s^2}}
            \p{-\rho_0 g}\nonumber\\
        &= \frac{\delta P}{c_s^2} + \xi_r \frac{N^2}{g^2}.
\end{align}
Then the radial component is straightforwardly
\begin{equation}
    \frac{1}{\rho_0}\pd{}{r}\delta P
        + \pd{}{r}\p{\delta \Phi}
        - \omega^2\xi_r
        + \frac{g}{\rho_0}\p{\frac{\delta P}{c_s^2} + \xi_r \frac{N^2\rho_0}{g}}
        = 0.
\end{equation}
Finally, the Poisson equation is simplified straightforwardly (with spherical
harmonic dependence) to
\begin{align}
    \nabla^2\p{\delta \Phi} &= 4\pi G \delta \rho,\nonumber\\
    \frac{1}{r^2}\pd{}{r}\p{r^2\pd{}{r}(\delta \Phi)}
        - \frac{l(l + 1)}{r^2}(\delta \Phi)
        &= 4\pi G \rho_0\p{\frac{\delta P}{\rho_0 c_s^2}
            + \xi_r\frac{N^2}{g}}.
\end{align}
Thus, the three equations for radial oscillations are a combination of: the
perpendicular momentum equation in combination with the density equation, the
radial momentum equation, and the Poisson equation. For posterity, they are
(with slight rearrangements):
\begin{align}
    \frac{1}{r^2}\pd{}{r}\p{r^2\xi_r}
            - \frac{g}{c_s^2}\xi_r
            + \p{1 - \frac{l(l + 1)c_s^2}{r^2\omega^2}}
                \frac{\delta P}{c_s^2\rho_0}
            - \frac{l(l + 1)}{r^2\omega^2}\delta \Phi
        &= 0,\label{eq:pert_dim1}\\
    \frac{1}{\rho_0}\pd{}{r}\delta P
        + \pd{}{r}\delta \Phi
        + \frac{g}{\rho_0c_s^2}\delta P
        + \p{N^2 - \omega^2}\xi_r
        &= 0,\label{eq:pert_dim2}\\
    \frac{1}{r^2}\pd{}{r}\p{r^2\pd{}{r}(\delta \Phi)}
        - \frac{l(l + 1)}{r^2}(\delta \Phi)
        - 4\pi G \rho_0
            \p{\frac{\delta P}{\rho_0 c_s^2} + \xi_r\frac{N^2}{g}} &= 0.
            \label{eq:pert_dim3}
\end{align}
This constitutes 3 ODEs for 3 quantities, but one is second order. To make a
reduction to first order, $\p{\delta \Phi}'$ constitutes a fourth variable.

\subsection{Nondimensional Form}

To nondimensionalize, we use the variables recommended in Unno's 1979 book (and
by Le Bihan). They are:
\begin{align}
    y_1 &= \frac{\xi_r}{r},\\
    y_2 &= \frac{1}{gr}\p{\frac{\delta P}{\rho_0} + \delta \Phi},\\
    y_3 &= \frac{1}{gr}\delta \Phi,\\
    y_4 &= \frac{1}{g} \rd{}{r}\delta \Phi,\\
    x &= \frac{r}{R}.
\end{align}
To make this substitution, it helps not to use the fully-simplified forms
initially. For instance, to simplify Eq.~\eqref{eq:pert_dim1}, it helps to start
with the earlier form (primes denote $\partial_r$; we will explicitly denote $x$
derivatives)
\begin{align}
    -\frac{\delta P}{c_s^2\rho_0} + \frac{\xi_r g}{c_s^2}
        &= \frac{1}{r^2}\pd{}{r}(r^2\xi_r)
        - \frac{l(l + 1)}{r^2 \omega^2}\p{\frac{\delta P}{\rho_0} + \delta \Phi}
        ,\nonumber\\
    -\frac{1}{c_s^2}\p{gry_2 - \delta \Phi} + \frac{ry_1 g}{c_s^2}
        &= \frac{1}{r^2}\pd{}{r}(r^3y_1)
        - \frac{l(l + 1)}{r^2 \omega^2}gry_2
        ,\nonumber\\
    -\frac{gr}{c_s^2}\p{y_2 - y_3} + \frac{ry_1 g}{c_s^2}
        &= 3y_1 + ry_1'
        - \frac{l(l + 1)}{r^2 \omega^2}gry_2
        ,\nonumber\\
    ry_1' &= -\frac{gr}{c_s^2}\p{y_2 - y_3} + \frac{y_1 gr}{c_s^2}
            + \frac{l(l + 1)}{r^2 \omega^2}gry_2
            - 3y_1\nonumber\\
    xy_1'(x) &= \p{\frac{gr}{c_s^2} - 3}y_1
            + \s{\frac{l(l + 1)g}{r\omega^2} - \frac{gr}{c_s^2}}y_2
            + \frac{gr}{c_s^2}y_3.
\end{align}
To simplify Eq.~\eqref{eq:pert_dim2}, we do
\begin{align}
    \delta P &= (y_2 - y_3)gr\rho_0,\\
    0 &= \frac{1}{\rho_0}\pd{((y_2 - y_3)gr\rho_0)}{r}
            + \pd{\p{gry_2}}{r}
            + \frac{g(y_2 - y_3)gr\rho_0}{\rho_0c_s^2}
            + \p{N^2 - \omega^2}ry_1,\nonumber\\
        &= \frac{\rho_0'(y_2 - y_3)gr}{\rho_0}
            + \pd{\p{gry_2}}{r}
            + \frac{g(y_2 - y_3)gr\rho_0}{\rho_0c_s^2}
            + \p{N^2 - \omega^2}ry_1,\nonumber\\
    ry_2' &= -\frac{\rho_0'(y_2 - y_3)r}{\rho_0}
            - y_2\frac{(gr)'}{g}
            - \frac{(y_2 - y_3)gr}{c_s^2}
            - \frac{\p{N^2 - \omega^2}}{g}ry_1,\nonumber\\
    xy_2'(x) &= \frac{\p{\omega^2 - N^2}}{g}ry_1
            - \p{\frac{\rho_0' r}{\rho_0} + \frac{(gr)'}{g} + \frac{gr}{c_s^2}}y_2
            + \p{\frac{\rho_0' r}{\rho_0} + \frac{gr}{c_s^2}}y_3,\nonumber\\
    \frac{\rho_0' r}{\rho_0} &= \p{\frac{N^2}{g^2} + \frac{1}{c_s^2}}\p{-gr},\\
    xy_2'(x) &= \frac{\p{\omega^2 - N^2}}{g}ry_1
            + \p{\frac{N^2r}{g} - \frac{(gr)'}{g}}y_2
            - \frac{N^2r}{g}y_3,\nonumber\\
    \frac{1}{r^2}\pd{}{r}\p{r^2\pd{\Phi_0}{r}} &= 4\pi G \rho_0,\nonumber\\
    \pd{}{r}\p{r^2g} &= 4\pi G\rho_0r^2,\nonumber\\
    r(gr)' &= 4\pi G\rho_0r^2 - gr,\nonumber\\
    \frac{(gr)'}{g} &= \frac{4\pi \rho_0 r^3}{M_r} - 1,\\
    xy_2'(x) &= \s{\frac{\omega^2r}{g} - \frac{N^2r}{g}}y_1
            + \p{\frac{N^2r}{g} + 1 - \frac{4\pi \rho_0 r^3}{M_r}}y_2
            - \frac{N^2r}{g}y_3.
\end{align}
Eq.~\eqref{eq:pert_dim3} also follows:
\begin{align}
    0 &= \frac{1}{r^2}\p{r^2gy_4}'
            - \frac{l(l+1)}{r^2}gry_3
            - 4\pi G \rho_0\p{
                \frac{(y_2 - y_3)gr\rho_0}{\rho_0c_s^2}
                + ry_1\frac{N^2}{g}
            },\nonumber\\
    0 &= (gr^2)'\frac{y_4}{r} + gry_4'
            - \p{l(l+1)}gy_3
            - 4\pi G \rho_0r\p{
                \frac{(y_2 - y_3)gr}{c_s^2}
                + y_1\frac{N^2r}{g}
            },\nonumber\\
    ry_4' &= -(4\pi G\rho_0r^2)\frac{y_4}{gr}
            + \p{l(l+1)}y_3
            + \frac{4\pi G\rho_0 r}{g}
                \p{
                    \frac{(y_2 - y_3)gr}{c_s^2}
                    + y_1\frac{N^2r}{g}
                },\nonumber\\
        &= \frac{4\pi \rho_0r^3}{M_r}\s{-y_4
                + \frac{(y_2 - y_3) gr}{c_s^2}
                + \frac{N^2r}{g}y_1}
            + l(l+1)y_3.
\end{align}
Finally, the implicit equation is
\begin{align}
    (gry_3)' &= gy_4,\nonumber\\
    ry_3' &= y_4 - \frac{(gr)'}{g}y_3\nonumber\\
        &= y_4 - \s{\frac{4\pi \rho_0 r^3}{M_r} - 1}y_3.
\end{align}
Defining the constants
\begin{align}
    V_g &= \frac{gr}{c_s^2},\\
    U &= \frac{4\pi \rho_0 r^3}{M_r},\\
    c_1 &= \textcolor{red}{\frac{(r / R)^3}{M_r / M}},\\
    \bar{\omega}^2 &= \frac{\omega^2R^3}{GM},\\
    A^* &= \frac{N^2r}{g},
\end{align}
then we obtain
\begin{align}
    \frac{g}{r\omega^2} &= \frac{GM_r / r^2}{rGM\bar{\omega}^2/R^3}
        = \frac{M_r / M}{x^3}\frac{1}{\bar{\omega}^2}
        = \frac{1}{c_1\bar{\omega}^2},\nonumber\\
    xy_1' &= \p{V_g - 3}y_1 +
        \s{\frac{l(l+1)}{c_1\bar{\omega}^2}
            - V_g}y_2
        + V_gy_3,\\
    xy_2' &= \s{c_1\bar{\omega}^2 - A^*}y_1
        + \p{A^* + 1 - U}y_2
        \textcolor{red}{- A^* y_3},\\
    xy_3' &= \p{1 - U}y_3 + y_4,\\
    xy_4' &= UA^*y_1
        + UV_gy_2
        + \s{l(l+1) - UV_g}y_3
        \textcolor{red}{- Uy_4}.
\end{align}
The red parts differ from the Le Bihan paper. Notably, $c_1$ differs from the Le
Bihan paper, and we should obtain
\begin{equation}
    \textcolor{red}{c_1 = \frac{\p{r / R}^3}{M_r / M}
        = \frac{\xi / \xi_1}{\theta'(\xi) / \theta'(\xi_1)}}.
\end{equation}
Note that in this case, $c_1(\xi \to 0) = -3\theta'(\xi_1) / \xi_1$ is finite.
As such, we have fully constrained the eigenvalue problem for $\z{y_i(x),
\omega}$ in terms of the fields $V_g$, $c_1$, $A^*$, and $U$.

\subsection{Boundary Conditions / Behavior}

Near $x = 0$, $V_g, A^* \propto x^2$ while $U \approx 3$ and $c_1 \approx
-3\rdil{\theta(1)}{x}$. The inner BCs are standard (we read from Le Bihan, but
Fuller 2017 also seems to have similar):
\begin{align}
    0 &= \xi_r - \frac{l}{\omega^2 r}\p{\frac{\delta P}{\rho_0} + \delta \Phi}
        ,\\
    0 &= \rd{(\delta \Phi)}{r} - \frac{l}{r}\delta \Phi.
\end{align}
These nondimensionalize to
\begin{align}
    0 &= ry_1 - \frac{l}{\omega^2r}gry_2\nonumber\\
        &= y_1 - l\frac{GM_r / r^2}{rGM\bar{\omega}^2 / R^3}y_2\nonumber\\
        &= y_1 - \frac{l}{c_1\bar{\omega}^2}y_2,\\
    0 &= gy_4 - \frac{l}{r}gry_3\nonumber\\
        &= y_4 - ly_3.
\end{align}
In order then for $xy_3' \to 0$ as $x \to 0$, we require $xy_3' \propto
(l-2)y_3$ or that
\begin{equation}
    y_3(x \ll 1) \sim x^{l - 2}.
\end{equation}
In addition, for $xy_1' \to 0$, we obtain $-3 y_1 + (l + 1)y_1 = xy_1'$, so we
also find
\begin{equation}
    y_1(x \ll 1) \sim x^{l - 2}.
\end{equation}
What about the other two $y_i'$ equations? Well, $xy_2' = ly_2 - 2y_2$, so $y_2
\sim x^{l - 2}$ as well, and $xy_4' = (l+1)y_4 - 3y_4$, so $y_4 \sim x^{l - 2}$
at last. These two BCs constrain two DOF\@; the overall normalization is a third
DOF, and the ratio $y_3 / y_1$ is a fourth. So we choose two different $y_3 /
y_1$ in constructing our solutions.

As for the outer BCs, they are (small correction to Le Bihan; consistent with
Fuller)
\begin{align}
    0 &= \rd{(\delta \Phi)}{r} + \textcolor{red}{\frac{(l + 1)}{r}}\delta \Phi,\\
    0 &= \Delta P = \delta P - \xi_r g\rho_0.
\end{align}
Nondimensionalizing, we obtain
\begin{align}
    0 &= gy_4 + \frac{(l + 1)}{r}gry_3\nonumber\\
        &= y_4 + (l + 1)y_3,\\
    0 &= gr\rho_0 (y_2 - y_3) - gr\rho_0 y_1\nonumber\\
        &= y_1 - y_2 + y_3.
\end{align}
Near the surface, $V_g$ and $A^*$ diverge, $U \to 0$, and $c_1 \to 1$. First, we
note that $xy_3' = y_3 + y_4 = y_3 - (l + 1)y_3 = -ly_3$, and so
\begin{equation}
    y_3\p{x \to 1} \sim x^{-l}.
\end{equation}
For $y_1$, we look at the singular terms in $xy_1'$ and $xy_2'$, but $xy_1' =
V_g\p{y_1 - y_2 + y_3}$ and $xy_2' = -A^*\p{y_1 - y_2 + y_3}$ both identically
cancel (which is good, since the LHS doesn't have a divergence to cancel these
out!). So, doing a bit more work, the other equations are
\begin{align}
    xy_1' &= -3y_1 + \frac{l(l+1)}{\bar{\omega}^2}y_2,\\
    xy_2' &= \bar{\omega}^2y_1 + y_2,\\
    xy_4' &= UA^*y_1 + UV_g(y_2 + y_4 / (l + 1)) - ly_4.
\end{align}
Note that
\begin{equation}
    UV_g \propto UA^* = \frac{4\pi \rho_0 R^3}{M_r}\frac{GM_r/R}{c_s^2}
        \propto \frac{4\pi G\rho_c\theta^n(1) R^2}{\theta(1)}
        \propto \theta^{n - 1}(1).
\end{equation}
This is zero if $n > 1$. Thus, the last equation shows $y_4 \propto x^{-l}$ as
well. If we also want $y_1, y_2 \propto x^{p}$, then we need ($z \equiv
y_1\bar{\omega}^2$)
\begin{align}
    pz &= -3z + l(l+1)y_2,\\
    py_2 &= z + y_2,\\
    (p+3)(p-1) &= l(l+1).
\end{align}
This doesn't work! Thus, $y_1$ and $y_2$ have different power law behaviors at
the edge, so we should instead write
\begin{align}
    p_1z &= -3z + l(l+1)y_2,\\
    p_2y_2 &= z + y_2,\\
    (p_1+3)(p_2-1) &= l(l+1).
\end{align}
Then either $p_2 = l + 1, p_1 = l - 2$ or $p_2 = l + 2$ and $p_1 = l - 3$. Le
Bihan says that the correct one to use is
\begin{align}
    y_1 \sim x^{l - 2}, & \quad y_2 \sim x^{l + 1}.
\end{align}

\textbf{Note:} After correcting all the expressions and frequencies and checking
the BCs, it seems that I was passing in $x$ to an interpolation routine that
accepted $\xi$ instead\dots welp.

Also, \textbf{NOTE:} I changed the signs back to those in Le Bihan, and I get frequencies
that are closer to theirs. So could it be possible that they didn't just make a
typo, and have a different version of the Equations? TODO check the sympy and
make sure I have the right signs.

\end{document}

